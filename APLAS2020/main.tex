% This is samplepaper.tex, a sample chapter demonstrating the
% LLNCS macro package for Springer Computer Science proceedings;
% Version 2.20 of 2017/10/04
%
\documentclass[runningheads]{llncs}

\usepackage{stmaryrd}
\usepackage{amsmath}
\usepackage{amssymb}
\usepackage{common/common}
\usepackage{mathpartir}
\usepackage{listings}
\usepackage{color,soul}




%\definecolor{lightYellow}{rgb}{1,1,0.8}
\definecolor{lightYellow}{rgb}{1,1,0.8}



\lstdefinestyle{custom_lang}{
%   language = scala,
	escapeinside={*}{*},
  	showstringspaces=false,
  	basicstyle={\ttfamily},
  	keywordstyle=\color{blue},
  	keywords = [1]{def},
  	otherkeywords = {<:, >:, =, ->, =>},
  	morekeywords = [2]{<:, >:, =, ->, =>, this},
%	
	keywordstyle = [1]{\tt \bfseries \color{red}},
	keywordstyle = [2]{\tt \bfseries \color{blue}},
%	morekeywords = [2]{class, interface, module, val, type, def, object, var},
%	morekeywords = [3]{match, with, this},
%	morekeywords = [4]{<:, >:, =, ->, =>},
	frame=single
}
%
\usepackage{graphicx}
% Used for displaying a sample figure. If possible, figure files should
% be included in EPS format.
%
% If you use the hyperref package, please uncomment the following line
% to display URLs in blue roman font according to Springer's eBook style:
% \renewcommand\UrlFont{\color{blue}\rmfamily}

\begin{document}
%
\title{Contribution Title\thanks{Supported by organization x.}}
%
%\titlerunning{Abbreviated paper title}
% If the paper title is too long for the running head, you can set
% an abbreviated paper title here
%
\author{First Author\inst{1}\orcidID{0000-1111-2222-3333} \and
Second Author\inst{2,3}\orcidID{1111-2222-3333-4444} \and
Third Author\inst{3}\orcidID{2222--3333-4444-5555}}
%
\authorrunning{F. Author et al.}
% First names are abbreviated in the running head.
% If there are more than two authors, 'et al.' is used.
%
\institute{Princeton University, Princeton NJ 08544, USA \and
Springer Heidelberg, Tiergartenstr. 17, 69121 Heidelberg, Germany
\email{lncs@springer.com}\\
\url{http://www.springer.com/gp/computer-science/lncs} \and
ABC Institute, Rupert-Karls-University Heidelberg, Heidelberg, Germany\\
\email{\{abc,lncs\}@uni-heidelberg.de}}
%
\maketitle              % typeset the header of the contribution
%
\begin{abstract}
The abstract should briefly summarize the contents of the paper in
150--250 words.

\keywords{First keyword  \and Second keyword \and Another keyword.}
\end{abstract}

\section{Introduction}
\label{s:intro}

Bounded quantification (or bounded polymorphism) is a powerful and widely used language construct that introduces a form of 
abstraction for types. Where functions provide an abstraction of behavior for values, bounded quantification 
provides an abstraction of behavior for types. A motivating example is an addition function for numbers. Below we
provide such an example using no particular language syntax (although it resembles Scala).
\begin{lstlisting}[mathescape, style=custom_lang]
def plus [A <: Real] : A -> A -> A
\end{lstlisting}
The type \verb|A| is unimportant except in that A is some subtype of \verb|Real| (the upper bound on \verb|A|). Ideally we would like to 
\verb|plus| to be defined abstractly for any value that could be considered a \verb|Real|, and not have to write 
a separate function for \verb|Integer|, \verb|Natural|, and \verb|Real|. \verb|plus| is  quantified over type \verb|A| which is 
bounded by \verb|Real|.

Bounded quantification has been adopted by many different languages, and is not exclusive to any specific paradigm.
Java Generics provide a form of bounded quantification for both method and class definitions.
Scala includes generics similar to that of Java (only more flexible), but adds abstract type members on top, further complicating matters.

Unfortunately, several forms of bounded quantification have been shown to exhibit undecidable subtyping.
To the surprise of many at the time, Pierce \cite{fsub popl 92} demonstrated that subtyping in \Fsub, a typed $\lambda$-calculus with subtyping and bounded quantification, 
was undecidable by a reduction to the halting problem.
More recently , and to perhaps less surprise, subtyping of Java Generics was also shown to be undecidable.
Hu \cite{jason and andreij} showed subtyping of D$_{<:}$, a minimal calculus, capturing parts of the Scala type system, was undecidable by a reduction to an undecidable fragment of \Fsub.

If subtyping in languages with relatively wide usage is undecidable, then one might ask the question: how important is decidable subtyping in practice?
Unfortunately, undecidability means that type checking of certain programs will not terminate, and will eventually crash without any error message indicating the problem.
In writing a compiler, one fix to this problem might be to enforce a maximal depth on proof search, or to simply timeout during type checking. These are unsatisfying 
solutions, as not only might they exclude some false negatives, but they also won't be able to provide the programmer much guidance on debugging their program.
Thus, while presumably rare, the potential problems are severe.

In this paper we show how simple syntactic restrictions can allow for decidable forms of bounded quantification that are easy to type check, 
allow for informative error messages, all while retaining many of the useful properties of typing in \Fsub: subtype transitivity, type safety, and minimal typing.

\section{The Undecidability of Bounded Quantification in \Fsub}

Bounded polymorphism was formalized in \Fsub by Cardelli \cite{cardelli}, and shown to be undecidable by Pierce \cite{pierce popl 92}.
\Fsub introduces bounded quantification to the simply typed $\lambda$-calculus by way of a universally quantified syntactic form with the
following typing rule.
\begin{mathpar}
\infer
	{
	\Gamma, (\alpha \leqslant \tau_1) \vdash t\ :\ \tau_2
	}
	{
	\Gamma \vdash \Lambda (\alpha \leqslant \tau_1).t\ :\ \forall (\alpha \leqslant \tau_1).\tau_2
	}
\end{mathpar}
That is, term $t$, with type $\tau_2$, is quantified over some type, represented by $\alpha$, whose upper bound is $\tau_1$.
The undecidability of subtyping in \Fsub was demonstrated by a reduction of subtyping to the halting problem.
The reduction to the halting problem relies on the contra-variance in the subtyping rule for universally quantified types given below.
\begin{mathpar}
\infer
	{
	\Gamma \vdash \tau_2\ <:\ \tau_1 \\
	\Gamma, (\alpha \leqslant \tau_2) \vdash \tau_1'\ <:\ \tau_2'
	}
	{
	\Gamma \vdash \forall (\alpha \leqslant \tau_1).\tau_1'\ :\ \forall (\alpha \leqslant \tau_2).\tau_2'
	}
	\quad(\textsc{S-All})
\end{mathpar}
As can be seen above, subtyping of universally quantified types in \Fsub allows for contra-variance on the bound of the quantified type.
Kernel F$_{<:}$, a variant of \Fsub, has been shown to be decidable in its subtyping \cite{pierce}. Kernel F$_{<:}$ removes the contra-variance of the \textsc{S-All} rule above, 
and instead enforces invariance on the bound.
\begin{mathpar}
\infer
	{
	\Gamma, (\alpha \leqslant \tau) \vdash \tau_1\ <:\ \tau_2
	}
	{
	\Gamma \vdash \forall (\alpha \leqslant \tau).\tau_1\ :\ \forall (\alpha \leqslant \tau).\tau_2
	}
	\quad(\textsc{S-All-Kernel})
\end{mathpar}
While decidable, \textsc{S-All-Kernel} is unsatisfying as it excludes desirable behavior. 
Ideally, we would like the \verb|plus| function, from Section \ref{s:intro}, to be substitutable for some other function defined for the more general type \verb|Number|:
\begin{lstlisting}[mathescape, style=custom_lang]
def add [A <: Number] : A -> A -> A
\end{lstlisting}
Castagna and Pierce \cite{ftop stuff} attempted to introduce such variance in a safe way by proposing F$^\top_{<:}$ 
with the following subtyping rule for universally quantified types.
\begin{mathpar}
\infer
	{
	\Gamma \vdash \tau_2\ <:\ \tau_1 \\
	\Gamma, (\alpha \leqslant \top) \vdash \tau_1'\ <:\ \tau_2'
	}
	{
	\Gamma \vdash \forall (\alpha \leqslant \tau_1).\tau_1'\ :\ \forall (\alpha \leqslant \tau_2).\tau_2'
	}
	\quad(\textsc{S-All$^\top$})
\end{mathpar}
Unfortunately, while decidable, F$_{<:}^\top$ sacrifices minimal typing. 
That is, it is possible to write a term in F$_{<:}^\top$ that can be typed 
with two different, and unrelated types \cite{ftop autopsy}. A lack of minimal 
typing means that the typing algorithm for F$_{<:}^\top$ is not complete.

\section{Constrained Contra-variance in \Fsub}
In this section we present a variant of \Fsub that introduces a syntactic restriction on bounded quantification to achieve decidable subtyping.
We start by introducing the type syntax of \Fsub in Figure \ref{f:fsub :syntax}. Since we are only concerned with subtyping, and not typing, we only present 
the type syntax. The term syntax and typing rules can be found in Appendix \ref{app:fsub}. 
A type in \Fsub is either the top type ($\top$), a bounded type variable ($\alpha$), an arrow type ($\tau \rightarrow \tau$), or a universally quantified type ($\forall(\alpha\leqslant\tau).\tau$).

\begin{figure}[t]
\begin{minipage}{\linewidth}
\[
%\begin{array}{lcr}
\begin{syntax}
\syntaxElement{\tau}{\Fsub Type}
	{\top}{top}
	{\alpha}{variable}
	{\tau \rightarrow \tau}{arrow}
	{\forall (\alpha \leqslant \tau).\tau}{all}
\endSyntaxElement
\end{syntax}
\]
\caption{\Fsub Type Syntax}
\label{f:fsub :syntax}
\end{minipage}
\end{figure}

In Figure \ref{f:fsub:subtyping} we define the subtyping of F$_{<:}^N$, a normal form of subtyping in \Fsub, defined by Pierce \cite{popl 92}. Subtyping is bounded above by $\top$ (\textsc{S$^N$-Top}) and explicitly reflexive in the case of type variables (\textsc{S$^N$-Rfl}).
A type supertypes a type variable if it supertypes its upper bound (\textsc{S$^N$-Var}).
Subtyping of arrow types is contra-variant with respect to its argument type, and covariant with respect to its return type (\textsc{S$^N$-Arr}).
Finally, subtyping of universally quantified types is contra-variant with respect to the type bounds, and covariant with respect to the type bodies.
\begin{figure}[t]
\begin{mathpar}
\infer
	{}
	{\Gamma \vdash \tau\ <:\ \top}
	\quad (\textsc{S$^N$-Top})
	\and
\infer
	{}
	{\Gamma \vdash \alpha\ <:\ \alpha}
	\quad (\textsc{S$^N$-Rfl})
	\and
\infer
	{
	(\alpha\ \leqslant\ \tau)\ \in\ \Gamma \\
	\Gamma \vdash \tau'\ <: \tau
	}
	{\Gamma \vdash \alpha\ <:\ \tau}
	\quad (\textsc{S$^N$-Var})
	\and
\infer
	{
	\Gamma \vdash \tau_2\ <:\ \tau_1 \\
	\Gamma  \vdash \tau_1'\ <:\ \tau_2'
	}
	{\Gamma \vdash \tau_1 \rightarrow \tau_1'\ <:\ \tau_2 \rightarrow \tau_2'}
	\quad (\textsc{S$^N$-Arr})
	\and
\infer
	{
	\Gamma \vdash \tau_2\ <:\ \tau_1 \\
	\Gamma, (\alpha \leqslant \tau_2) \vdash \tau_1'\ <:\ \tau_2'
	}
	{\Gamma \vdash \forall (\alpha\leqslant\tau_1).\tau_1'\ <:\ \forall(\alpha \leqslant \tau_2).\tau_2'}
	\quad (\textsc{S$^N$-All})
\end{mathpar}
\caption{\Fsub Subtyping}
\label{f:fsub:subtyping}
\end{figure}

Achieving a decidable variant of \Fsub follows a simple idea: we restrict contra-variance of type bounds to only types that do not themselves contain bounded quantification.
In Figure \ref{f:restricted:syntax}  we introduce a separated syntax for the syntax of \Fsub. 
\begin{figure}[t]
\begin{minipage}{\linewidth}
\[
%\begin{array}{lcr}
\begin{syntax}
\syntaxElement{\tau}{Separated \Fsub Type}
	{\top}{top}
	{\alpha}{variable}
	{\gamma}{restricted variable}
	{\tau \rightarrow \tau}{arrow}
	{\forall (\gamma \leqslant \mu).\tau}{restricted all}
	{\forall (\delta \leqslant \tau).\tau}{all}
\endSyntaxElement\\
\syntaxElement{\alpha}{Type Variable}
	{\gamma}{unrestricted}
	{\delta}{restricted}
\endSyntaxElement\\
\syntaxElement{\mu}{Restricted Type}
	{\top}{top}
	{\gamma}{variable}
	{\mu \rightarrow \mu}{arrow}
	{\red{\forall (\gamma \leqslant \top).\mu}}{\red{all???}}
\endSyntaxElement
\end{syntax}
\]
\caption{A Separated \Fsub Type Syntax}
\label{f:restricted:syntax}
\end{minipage}
\end{figure}
Types containing no bounded quantification are identifed by $\mu$. Their only difference from more general types is a lack of bounded quantification.
A restricted type, $\gamma$, is either $\top$, a type variable, or an arrow type.
We now define a restricted subtyping relation using the rule set in Figure \ref{f:sep:subtyping} (Note: we omit rules for subtyping of $\top$, reflexivity, type variables, and arrow types as they are unchanged from Figure \ref{f:fsub:subtyping}).
\begin{figure}[t]
\begin{mathpar}
\infer
	{
	\Gamma, (\alpha \leqslant \tau)  \vdash \tau_1\ <:^R\ \tau_2
	}
	{\Gamma \vdash \forall(\alpha \leqslant \tau).\tau_1\ <:^R\ \forall(\alpha \leqslant \tau).\tau_2}
	\quad (\textsc{S$^R$-All-Kernel})
	\and
\infer
	{
	\Gamma \vdash \mu_2\ <:^R\ \mu_1 \\
	\Gamma, (\gamma \leqslant \mu_2) \vdash \tau_1\ <:^R\ \tau_2
	}
	{\Gamma \vdash \forall (\gamma\leqslant\mu_1).\tau_1\ <:^R\ \forall(\gamma \leqslant \mu_2).\tau_2}
	\quad (\textsc{S$^R$-All})
\end{mathpar}
\caption{Separated \Fsub Subtyping}
\label{f:sep:subtyping}
\end{figure}

The subtyping defined in Figure \ref{f:sep:subtyping} replaces the \textsc{S$^N$-All} rule with two rules: \textsc{S$^R$-All-Kernel} and \textsc{S$^R$-All}. 
\textsc{S$^R$-All-Kernel} is exactly the rule for subtyping of bounded quantification found in Kernel F$_{<:}$, that is, 
for $\forall(\alpha \leqslant \tau_1).\tau_1'$ to subtype $\forall(\alpha \leqslant \tau_2).\tau_2'$, $\tau_1$ and $\tau_2$ must be syntactically equivalent.
Contra-variance is allowed only in cases where the type bounds are of the form $\gamma \leqslant \mu$, and thus do not themselves include bounded quantification.
This is captured by the rule \textsc{S$^R$-All}.

The result of this restriction is that subtyping may only introduce new quantified types into the context if they are common to both types.

\subsection{Subtype Decidability}

\subsection{Subtype Transitivity}

\subsection{Type Safety}




%\section{Relaxing the Constraints on \Fsub}
%
%
%
%
%
%\begin{figure}[t]
%\begin{minipage}{\linewidth}
%\[
%%\begin{array}{lcr}
%\begin{syntax}
%\syntaxElement{\gamma}{System F Type}
%	{\top}{top}
%	{\alpha}{variable}
%	{\gamma \rightarrow \gamma}{arrow}
%	{\forall \alpha.\tau}{all}
%\endSyntaxElement
%\end{syntax}
%\]
%\caption{System F Type Syntax}
%\label{f:sysf:syntax}
%\end{minipage}
%\end{figure}

\section{Separating D$_{<:}$}

D$_{<:}$ is a calculus related to \Fsub that includes abstract type members and dependent functions, and serves to model core aspects of 
the Scala type system. The syntax of D$_{<:}$ is given in Figure \ref{f:dsub:syntax}, and at first glance does not immediately resemble 
that of \Fsub. Most noticably, D$_{<:}$ does not include any type variables. The expressiveness of D$_{<:}$ derives from being able to capture
\Fsub using  path types ($x.L$) and dependent function types ($\allType{x}{\tau_1}{\tau_2}$). 

\begin{figure}[t]
\begin{minipage}{\linewidth}
\[
\begin{syntax}
\syntaxElement{\tau}{D$_{<:}$ Type}
	{\top}{top}
	{\bot}{bottom}
	{\nonRec{}{\typeDef{L}{\tau}{\tau}}}{declaration}
	{x.L}{selection}
	{\allType{x}{\tau}{\tau^x}}{function}
\endSyntaxElement 
\end{syntax}
\]
\caption{D$_{<:}$ Type Syntax}
\label{f:dsub:syntax}
\end{minipage}
\end{figure}
A type in D$_{<:}$  is either $\top$, $\bot$, a type declaration $\nonRec{}{\typeDef{L}{\tau_1}{\tau_2}}$, a selection type $x.L$, or a dependent function type $\allType{x}{\tau_1}{\tau_2^x}$.

In D$_{<:}$, type declarations ($\nonRec{}{\typeDef{L}{\tau_1}{\tau_2}}$) define a type. Given a path $x$ to the type definition, 
the defined type can be used by selection on the path: $x.L$. That is, if value $x$ has type $\nonRec{}{\typeDef{L}{\tau_1}{\tau_2}}$, then $x.L$ 
refers to the defined type, where $\tau_1$ is the lower bound, and $\tau_2$ the upper bound.
This is useful when combined with the dependent function types of D$_{<:}$.
The return types of functions in D$_{<:}$  can be dependent on the argument. This dependence is indicated in the syntax by a super-script of the 
variable identifying the argument. i.e. $\allType{x}{\tau_1}{\tau_2^x}$ indicates that x is free in $\tau_2$.

\begin{figure}[t]
\small
\begin{mathpar}
\infer
	{}
	{\Gamma  \vdash \tau\ <:\ \top}
	\quad (\textsc{Top})
	\and
\infer
	{}
	{\Gamma  \vdash \tau\ <:\ \tau}
	\quad (\textsc{Rfl})
	\and
\infer
	{
	\Gamma(x) = \nonRec{}{\typeDef{L}{\tau_1}{\tau_2}}
	}
	{\Gamma  \vdash x.L\ <:\ \tau_2}
	\quad (\textsc{Sel1})
	\and
\infer
	{
	\Gamma(x) = \nonRec{}{\typeDef{L}{\tau_1}{\tau_2}}
	}
	{\Gamma  \vdash \tau_1\ <:\ x.L}
	\quad (\textsc{Sel2})
	\and
\infer
	{
	\Gamma \vdash \tau_1\ <:\ \tau_2 \\
	\Gamma \vdash \tau_2\ <:\ \tau_3
	}
	{\Gamma  \vdash \tau_1 \ <:\ \tau_3}
	\quad (\textsc{Trans})
	\and
\infer
	{
	\Gamma \vdash \tau_2\ <:\ \tau_1 \\
	\Gamma \vdash \tau_1'\ <:\ \tau_2'
	}
	{\Gamma  \vdash \nonRec{}{\typeDef{L}{\tau_1}{\tau_1'}} \ <:\ \nonRec{}{\typeDef{L}{\tau_2}{\tau_2'}}}
	\quad (\textsc{Bnd})
	\and
\infer
	{
	\Gamma \vdash \tau_2\ <:\ \tau_1 \\
	\Gamma, (x : \tau_2) \vdash \tau_1'\ <:\ \tau_2'
	}
	{\Gamma  \vdash \allType{x}{\tau_1}{\tau_1'} \ <:\ \allType{x}{\tau_2}{\tau_2'}}
	\quad (\textsc{All})
\end{mathpar}
\caption{D$_{<:}$ Subtyping}
\label{f:dsub:subtyping}
\end{figure}
D$_{<:}$ subtyping is defined in Figure \ref{f:dsub:subtyping}. Subtyping is bound above by $\top$ (\textsc{Top}) and below by $\bot$ (\textsc{Bot}).
Subtyping is explicitly reflexive (\textsc{Rfl}). Selection types subtype their upper bounds (\textsc{Sel1}), and super type their lower bounds (\textsc{Sel2}).
Subtyping of type declarations are contra-variant with respect to the lower bounds, and covariant with respect to their upper bounds (\textsc{Bnd}).
Finally, subtyping of dependent fuction types is contra-variant with respect to the arument types, and covariant with respect to the return types, 
with the return types being dependent on the arguments (\textsc{All}).

Coupling type declarations together with dependent function types allows for similar functionality to F$_{<:}$. That is, we can use the encoding below to capture 
bounded quantification from \Fsub in D$_{<:}$.
\[
\begin{array}{lcl}
\llbracket\top\rrbracket 
		& \triangleq 
		& \top\\
\llbracket\alpha\rrbracket 
		& \triangleq 
		& x_\alpha.A\\
\llbracket\tau_1 \rightarrow \tau_2\rrbracket 
		& \triangleq 
		& \allType{x_\alpha}{\llbracket\tau_1\rrbracket}{\llbracket\tau_2\rrbracket^{x_\alpha}}\\
\llbracket\forall(\alpha\leqslant\tau_1).\tau_2\rrbracket 
		& \triangleq 
		& \allType{x_\alpha}{\nonRec{}{\typeDef{A}{\bot}{\llbracket\tau_1\rrbracket}}}{\llbracket\tau_2\rrbracket^{x_\alpha}}
\end{array}
\]
The above encoding is in fact not enough to demonstrate the undecidability of D$_{<:}$ due to the fact that while the following holds:
\[
\Gamma \vdash \tau_1\ <:\ \tau_2\ \Rightarrow\ \llbracket\Gamma\rrbracket \vdash \llbracket\tau_1\rrbracket\ <:\ \llbracket\tau_2\rrbracket
\]
the inverse does not.
\[
\llbracket\Gamma\rrbracket \vdash \llbracket\tau_1\rrbracket\ <:\ \llbracket\tau_2\rrbracket\ \not\Rightarrow\ \Gamma \vdash \tau_1\ <:\ \tau_2
\]
The reasons for this are due to the fact that functions in \Fsub are unrelated to universally quantified types, 
but in D$_{<:}$ they are both captured using dependent function types. The full proof for this was demonstrated 
by Hu \cite{dsub decidability}, but this does not affect the undecidability result of D$_{<:}$, as the proof of undecidability 
in \Fsub does not rely on function types.

In order to achieve a decidable variant of D$_{<:}$, we can apply the same principles as we did with \Fsub.
Figure \ref{f:dsub:separated:syntax} presents a separated syntax for D$_{<:}$.
\begin{figure}[t]
\begin{minipage}{\linewidth}
\[
\begin{syntax}
\syntaxElement{L}{Type Label}
	{M}{unrestricted}
	{S}{restricted}
\endsyntaxElement\\
\syntaxElement{\tau}{Separated D$_{<:}$ Type}
	{\top}{top}
	{\bot}{bottom}
	{\nonRec{}{\typeDef{L}{\tau}{\tau}}}{declaration}
	{x.L}{selection}
	{\allType{x}{\tau}{\tau^x}}{function}
\endSyntaxElement  \\
\syntaxElement{\tau}{Restricted D$_{<:}$ Type}
	{\top}{top}
	{\bot}{bottom}
	{\nonRec{}{\typeDef{L}{\tau}{\tau}}}{declaration}
	{x.M}{selection}
	{\allType{x}{\tau}{\tau}}{function}
	{\allType{x}{\tau}{\tau^x}}{dependent function}
\endSyntaxElement  \\
\end{syntax}
\]
\caption{D$_{<:}$ Type Syntax}
\label{f:dsub:separated:syntax}
\end{minipage}
\end{figure}




\section{Related Work}

\section{Conclusion}

%%
%% The next two lines define the bibliography style to be used, and
%% the bibliography file.
\bibliographystyle{ACM-Reference-Format}
\bibliography{sample-base}

%%
%% If your work has an appendix, this is the place to put it.
\appendix

\section{Proof}

\subsection{Full \Fsub Syntax and Typing}
\label{app:fsub}

\subsection{Subtype Decidability}

\subsection{Transitivity}

\subsection{Type Safety}

%
% ---- Bibliography ----
%
% BibTeX users should specify bibliography style 'splncs04'.
% References will then be sorted and formatted in the correct style.
%
% \bibliographystyle{splncs04}
% \bibliography{mybibliography}
%
\begin{thebibliography}{8}
\end{thebibliography}
\end{document}
